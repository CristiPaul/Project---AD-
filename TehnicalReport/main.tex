\documentclass{report}
\usepackage[utf8]{inputenc}

\title{Topological Sort}
\author{Dicu Cristian Paul \\ CEN 1.1B (Dobby)}
\date{Year 2017- 2018 \\ Semester II}

\usepackage{natbib}
\usepackage{graphicx}
\usepackage{algorithm2e}
\usepackage{algpseudocode}


\begin{document}

\maketitle

\chapter*{Problem Statement}
Topological sort. Implement two different algorithms to determine the minimun path between two vertexes ityn weighted directed graphs, e.g., Moore and Dijkstra


\chapter*{Application Design. For the first Method}
The algorithm uses a number of functions that represent the process they are used for: 
\begin{itemize}
    \item Function for indegree.
    \item Function for queues, checking if its empty, inserting, and deleting queue.
    \item Function for creating a graph.
\end {itemize}
\section*{The input.}
In the input is required to enter the number of vertices (example: 6), then the edges (example: 0 1). After writing  all the edges that you want, you can type "-1 -1" in order to stop the writing edges and show you the result.
\section*{The output.}
In the output, if the edges are correct than the algorithm will show the vertices in topological order.
\\Example:
\\Number of vertices: 6
\\Enter edge : 0 1
\\Enter edge : 0 2
\\Enter edge : 0 3
\\Enter edge : 0 5
\\Enter edge : 1 2
\\Enter edge : 1 4 
\\Enter edge : -1 -1
\\Vertices in topological order are : 0 1 3 5 2 4

\section*{How the algorithm works.}
For this method I used indegree array .
\\
\\The function insert queue verifies if the queue is initially empty, and also if the queue reaches the max, if it is it will print the message : Queue Overflow .
\\

\begin{algorithm}[H]
    \KwData{The queue:}
    \KwResult{Inserts if needed to queue:}
    {
        \eIf{queue is full}{
            print queue is full\;
        }{
          \If{queue is initially empty}{ 
                print add to queue\;
             }
      }  
    }
    \caption{Inserting in queue.}
\end{algorithm}

\\The function create graph is where all the inputs are. If the origin or the destination is greater than n ( which is the number of vertices) than it will show the message "Invalid edge!", but it will move on to the next input.
\\

\begin{algorithm}[H]
    \KwData{The graph:}
    \KwResult{The input, number of vertices, and the edges:}
    \ForAll{max edges}{
            print origin and destination\;
            \If{origin and destination == -1}{
                 break\;
                 }
            \If{origin and destination greater than the number or vertices}{
            print invalid edge\;
        }
    }
    \caption{Creating the graph.}
\end{algorithm}

\\The function indegree calculates the indegree, which is the number of edges directed into a vertex in a directed graph.
\\

\begin{algorithm}[H]
    \KwData{The graph}
    \KwResult{Prints the indegree}
    \ForAll{every vertice in the graph}{
        \If{adjacency matrix == 1}{
            integree ++\;
        }
    }
    \caption{Finding the Indegree.}
\end{algorithm}

\\The main function is where I used the indegree, using a for to find the indegree of each vertex, and a while where it adds verteces (v) to the topological array (topo order), after that  delete all the edges going from vertex (v).

\begin{algorithm}[H]
    \KwData{The graph}
    \KwResult{Prints the vertices in topological order}
     \ForAll{find the integree of each vertex}{
             }
     \While{the queue is not empty}{
            add vertex to topological order array\;
            \ForAll{delete all edges going from vertex}
             } 
     \If{count greater than the number of vertices}{
            print no topological ordering possible\;
             }
     
     \ForAll{vertices in topological order are}{
            print all vertices in topological order\;
            }
    \caption{Printing the board.}
\end{algorithm}

\chapter*{Application Design. For the secound Method.}

The algorithm uses a number of functions that represent the process they are used for :
\begin{itemize}
   \item Function for the topological sort.
   \item Function for initialize list, adjacent lists, and degrees.
   \item Function for entries adjacent lists.
   \item Function for creat adjacent lists.
\end{itemize}

\section*{The input}
In the input is required to enter the number of vertices(example 4), then the edges (example vertex 5 has edges 4 and 3 : 5 4 3).

\section*{The output}

It should enter the vertices in topological order.

\section*{How the program works}

For this method I used a weird algorithm, which I am pretty sure its not the best.
\\\\
I used a function called topological sort which is supposed to calculate the topological sort of the lists.
\\\\
The function entries adjacent list is where the input takes place, using a while which verifies if the vertices are greater than the counter.



























\end{document}